\documentclass{homework}

% ----- Students: How to use this file to prepare your solutions. -----
%
% First, ensure "homework.cls" is under the same directory as this file.
%
% Next, comment out the three lines of
%     \This is ... posted ... due on ...
% and use
%     \ThisSolution is for ... by ... on ...
% instead.
% If you use "on today.", the date on your computer is used.
% Otherwise, you can specify a particular date.
%
% An example is provided in the comments below.
%
% Next, for each "\Task ... -- ... (... points)":
% 1. DO NOT remove "\Task ... -- ... (... points)".
% 2. DO remove the task/subtask description text,
%    including removing "\begin{subtasks} ... \end{subtasks}".
%    They can be removed after you have written down your solutions,
%    as you might want to make use of the equations from the description.
% 3. After "\Task ... -- ... (... points)", use
%        "\Subtask{...} ..."
%    to specify the subtask number and write down your solutions.
%
% An example is provided in the comments below.
%
% "\This" and "\ThisSolution" can be used only once.
% "\This", "\ThisSolution", and "\Task" control sequences have
% special formats that MUST be respected (or your TeX compiler
% will complain).
% The prepositions and the final full stop MUST NOT be altered.
%
% ---------------------------------------------------------------------

\def\Gen{{\mathsf{Gen}}}
\def\Kg{{\mathsf{Kg}}}
\def\Enc{{\mathsf{Enc}}}
\def\Dec{{\mathsf{Dec}}}
\def\Perms{{\mathsf{Perms}}}
\def\Init{{\mathsf{Init}}}
\def\Eval{{\mathsf{Eval}}}
\def\AES{{\mathsf{AES}}}
\def\RP{{\mathsf{RP}}}
\def\BC{{\mathsf{BC}}}
\def\Adv{{\mathsf{Adv}}}
\def\prp{{\textup{prp}}}

\begin{document}

\This is Homework 1,
posted on Wednesday, October 4, 2023 at 11:59pm
and due on Wednesday,  October 11, 2023 at 11:59pm.

\section*{Instructions and Rules}
\begin{itemize}
\item Write your solutions clearly, and ideally, type them up.
If they are handwritten, you are responsible to ensure they are readable.
Justify \emph{all claims} of your solution.
Partially incorrect solutions can still be worth several points,
but unjustified incorrect solutions will result in zero points for the corresponding question.
\item You are not allowed to copy or transcribe answers to homework assignments from others or other sources. You are not allowed to look up solutions online.
\item You need to provide an individual solution.
You are allowed to have high-level discussions with other students
(for instance, review the definition of a concept,
discuss what a homework question mean, and
high-level approaches).
Please disclose if possible who you discussed with.
\item You are given \emph{six} late days with no question asked,
but can only use at most \emph{three} per homework.
To use late days, send an e-mail to
\href{mailto:cse426-staff@cs.washington.edu}%
{\texttt{\small cse426-staff@cs.washington.edu}}
\emph{before} the deadline.
Other late submissions will be considered on a per-case basis,
but expect to provide an explanation.
\end{itemize}

\Task 1 -- Encryption Scheme (10 points)

Let ${\doubleZ_{10}=\{0,1,\dots,9\}}$.
We consider a symmetric encryption scheme ${\Pi=(\Kg,\Enc,\Dec)}$, for which both the message and ciphertext spaces are
${\scriptM=\scriptC=\doubleZ_{10}^4}$, i.e.,
both a plaintext~$M$ and a ciphertext~$C$ consist of four decimal digits, and where:
\begin{itemize}
\item $\Kg$ outputs a secret key $K = (d,\pi)$, where
${d\draws\doubleZ_{10}}$ and ${\pi\draws\Perms(\doubleZ_{10})}$,
i.e.,
$d$ is a uniformly chosen random decimal digit and
$\pi$ is a uniformly chosen random permutation of the decimal digits.
\item 
The encryption algorithm is defined by the following procedure:
\begin{align*}
&\underline{\textbf{procedure }\Enc(K=(d,\pi),M=(M[1],\dots,M[4])):}\\
&x_0\gets d\\
&\textbf{for }i=1\textup{ to }4\textbf{ do}\\
&\qquad x_i\gets(x_{i-1}+M[i]+1 - i)\bmod 10\\
&\qquad C[i]\gets\pi(x_i)\\
&\textbf{return }C=(C[1],\dots,C[4])
\end{align*}
\end{itemize}
\begin{subtasks}
\item\points{4}
Complete the description of $\Pi$ by giving a decryption algorithm $\Dec$
that satisfies the correctness requirement discussed in class.
\item\points{6}
Show that this encryption scheme is not perfectly secret.
\end{subtasks}

% The first part of the file should look like this
% after you type your solution:

% \Task 1 -- Encryption Scheme (10 points)
%
% \Subtask{a}
%
% The decryption algorithm lorem ipsum.
%
% \Subtask{b}
%
% Perfectly secure or not perfectly secure,
% that is the question!

\Task 2 -- The Shuffle (19 points)

Consider the following symmetric encryption scheme $\Pi' =(\Kg',\Enc',\Dec')$ with plaintext space $\scriptM = \bit^n$. Moreover:
\begin{itemize}
\item $\Kg'$ outputs a secret key $\pi$, where
${\pi\draws\Perms(\{1, \ldots, 2n\})}$, i.e.,
$\pi$ is a uniformly chosen random permutation of the set $\{1, \ldots, 2n\}$.
\item 
The encryption algorithm is defined by the following procedure:
\begin{align*}
&\underline{\textbf{procedure }\Enc'(\pi,M=(M[1],\dots,M[n])):}\\
&M' \gets M \concat \overline{M}\\
&\textbf{for }i=1\textup{ to }2n\textbf{ do}\\
&\qquad C[i]\gets M'[\pi(i)]\\
&\textbf{return }C=(C[1],\dots,C[2n])
\end{align*}
Here, $\overline{M}$ is the bit-wise complement of $M$, i.e., $\overline{M}[i] = 1 - M[i]$ for all $i=1, \ldots, n$. Further, $M \concat \overline{M}$ is the concatenation of $M$ and $\overline{M}$.
\end{itemize}
\begin{subtasks}
\item \points{4} Describe the ciphertext space $\scriptC$, i.e., the set of all possible valid ciphertexts resulting from encrypting a plaintext $M \in \scriptM$ with some key.

\textcolorbf{Hint:} Find an invariant satisfied by $M'$ for all $M \in \bit^{n}$.


\item \points{4} Complete the description of $\Pi'$ by giving a decryption algorithm $\Dec'$
that satisfies the correctness requirement discussed in class.
\item \points{8} Characterize the distribution of $\Enc'(\pi, M)$, for a uniformly chosen $\pi$ and an arbitrary $M \in \bit^n$. 

\item \points{3} Conclude that $\Pi'$ satifies perfect secrecy.
\end{subtasks}





\Task 3 -- Playing with AES (10 points)

We want to develop a better sense of
the pseudorandomness of the ciphertexts generated by the AES block cipher.
In particular,
we will focus on the most commonly used variant with $128$-bit keys.
Let $X$ be the $16$-byte string
\begin{align*}
X = 10\; 04\; 20\; 18\;
00\; 00\; 00\; 00\; 00\; 00\; 00\; 00\; 00\; 00\; 00\; 00
\end{align*}
in hexadecimal format.

\begin{subtasks}
\item\points{2}
What is the value of $\AES(X, X)$?
Write the result in hexadecimal format.
Here, $\AES(K, M)$ is the ciphertext generated by AES on key~$K$ and block~$M$.
\item\points{4}
Find a $16$-byte block~$M$
such that the lower half of ${C = \AES(X, M)}$ is all zero.
In other words, $C$ ends with $00\; 00\; 00\; 00\; 00\; 00\; 00\; 00$.
Explain how you have found it! 
\item\points{4} Find a $16$-byte key $K$ with the property that the last
byte of $C = \AES(K, X)$ is equal to $00$.
Explain how you have found it!
\end{subtasks}
You can use the Python code for AES (\texttt{hw1.py}) provided on
\href{https://edstem.org/us/courses/47519/resources}{Ed},
or any of your favorite programming languages and libraries,
to help performing AES evaluations.  (Do {\em not} re-implement AES!)

\Task 4 -- Distinguishing Advantage (6 points)

The goal of this task is to practice with the notion of distinguishing
advantage.
%
\smallskip

\noindent To this end, we are given the following two oracles, $\mathsf{O}_0$ and~$\mathsf{O}_1$.
They both are initialized by running the (private) procedure $\Init()$, and
the adversary can then only call the procedure $\Eval()$.
\begin{center}
\begin{tabular}{|l|l|}
\hline
\underline{\textbf{oracle }$\mathsf{O}_0$:}\rule[1.2em]{0pt}{0pt} &
\underline{\textbf{oracle }$\mathsf{O}_1$:}\rule[1.2em]{0pt}{0pt} \\
& \\
\underline{\emph{private }\textbf{procedure }$\Init()$:} &
\underline{\emph{private }\textbf{procedure }$\Init()$:} \\
${b_1\draws\bit},\,{b_2\draws\bit}$ &
${b_1\draws\bit}$ \\
&
\textbf{if }${b_1=0}$\textbf{ then }${b_2\draws\bit}$ \\
&
\textbf{else }${b_2\gets 0}$ \\
& \\
\underline{\emph{public }\textbf{procedure }$\Eval()$:} &
\underline{\emph{public }\textbf{procedure }$\Eval()$:} \\
\textbf{return }${b_1\concat b_2}$\rule[-0.6em]{0pt}{0pt} &
\textbf{return }${b_1\concat b_2}$\rule[-0.6em]{0pt}{0pt} \\
\hline
\end{tabular}
\end{center}
Here, ${b_1\concat b_2}$ is the concatenation of $b_1$ and~$b_2$.
Consider the following distinguishers $D_1$ and~$D_2$,
which are given access to an oracle $\mathsf{O}$
that is either $\mathsf{O}_0$ or~$\mathsf{O}_1$:
\begin{center}
\begin{tabular}{|l|l|}
\hline
\underline{\textbf{distinguisher }$D_1^{\mathsf{O}}$:}\rule[1.2em]{0pt}{0pt} &
\underline{\textbf{distinguisher }$D_2^{\mathsf{O}}$:}\rule[1.2em]{0pt}{0pt} \\[4pt]
${b_1\concat b_2\gets \mathsf{O}.\Eval()}$\hspace*{5em} &
${b_1\concat b_2\gets \mathsf{O}.\Eval()}$\hspace*{5em} \\
\textbf{return }$b_1$\rule[-0.6em]{0pt}{0pt} &
\textbf{return }${b_1\oplus b_2}$\rule[-0.6em]{0pt}{0pt} \\
\hline
\end{tabular}
\end{center}
\begin{subtasks}
\item\points{3}
What is the advantage of~$D_1$
in distinguishing $\mathsf{O}_0$ and~$\mathsf{O}_1$?
\item\points{3}
What is the advantage of~$D_2$
in distinguishing $\mathsf{O}_0$ and~$\mathsf{O}_1$?
\end{subtasks}


\end{document}