%%%%%%%%%%%%%%%%%%%%% PACKAGE IMPORTS %%%%%%%%%%%%%%%%%%%%%
\documentclass[11pt]{article}
\usepackage{amsmath, amsfonts, amsthm, amssymb}
\usepackage{lmodern}
\usepackage{microtype}
\usepackage{fullpage}       
\usepackage{changepage}
\usepackage{hyperref}
\usepackage{blindtext}
\hypersetup{
    colorlinks=true,
    linkcolor=blue,
    filecolor=magenta,      
    urlcolor=blue,
    pdftitle={Overleaf Example},
    pdfpagemode=FullScreen,
    }
\urlstyle{same}

\newenvironment{level}%
{\addtolength{\itemindent}{2em}}%
{\addtolength{\itemindent}{-2em}}

\usepackage{amsmath,amsthm,amssymb}


\usepackage[x11names, rgb]{xcolor}
\usepackage{graphicx}
\usepackage[nooldvoltagedirection]{circuitikz}
\usetikzlibrary{decorations,arrows,shapes}

\usepackage{datetime}
\usepackage{etoolbox}
\usepackage{enumerate}
\usepackage{enumitem}
\usepackage{listings}
\usepackage{array}
\usepackage{varwidth}
\usepackage{tcolorbox}
\usepackage{amsmath}
\usepackage{circuitikz}
\usepackage{verbatim}
\usepackage[linguistics]{forest}
\usepackage{listings}
\usepackage{xcolor}
\renewcommand{\rmdefault}{cmss}


\newcommand\doubleplus{+\kern-1.3ex+\kern0.8ex}
\newcommand\mdoubleplus{\ensuremath{\mathbin{+\mkern-10mu+}}}

\definecolor{codegreen}{rgb}{0,0.6,0}
\definecolor{codegray}{rgb}{0.5,0.5,0.5}
\definecolor{codepurple}{rgb}{0.58,0,0.82}
\definecolor{backcolour}{rgb}{0.95,0.95,0.92}

\lstdefinelanguage{JavaScript}{
  keywords={typeof, new, true, false, catch, function, return, null, catch, switch, var, if, in, while, do, else, case, break},
  keywordstyle=\color{blue}\bfseries,
  ndkeywords={class, export, boolean, throw, implements, import, this},
  ndkeywordstyle=\color{darkgray}\bfseries,
  identifierstyle=\color{black},
  sensitive=false,
  comment=[l]{//},
  morecomment=[s]{/*}{*/},
  commentstyle=\color{purple}\ttfamily,
  stringstyle=\color{red}\ttfamily,
  morestring=[b]',
  morestring=[b]"
}

\lstdefinestyle{mystyle}{
    language=JavaScript,
    backgroundcolor=\color{backcolour},   
    commentstyle=\color{codegreen},
    keywordstyle=\color{magenta},
    numberstyle=\tiny\color{codegray},
    stringstyle=\color{codepurple},
    basicstyle=\ttfamily\footnotesize,
    breakatwhitespace=false,         
    breaklines=true,                 
    captionpos=b,                    
    keepspaces=true,                 
    numbers=left,                    
    numbersep=5pt,                  
    showspaces=false,                
    showstringspaces=false,
    showtabs=false,                  
    tabsize=2
}

\lstset{style=mystyle}
\setlength{\parindent}{0pt}
\setlength{\parskip}{5pt plus 1pt}

\providetoggle{questionnumbers}
\settoggle{questionnumbers}{true}
\newcommand{\noquestionnumbers}{
    \settoggle{questionnumbers}{false}
}

\newcounter{questionCounter}
\newenvironment{question}[2][\arabic{questionCounter}]{%
    \ifnum\value{questionCounter}=0 \else {\newpage}\fi%
    \setcounter{partCounter}{0}%
    \vspace{.25in} \hrule \vspace{0.5em}%
    \noindent{\bf \iftoggle{questionnumbers}{Question #1: }{}#2}%
    \addtocounter{questionCounter}{1}%
    \vspace{0.8em} \hrule \vspace{.10in}%
}

\newcounter{partCounter}[questionCounter]
\renewenvironment{part}[1][\alph{partCounter}]{%
    \addtocounter{partCounter}{1}%
    \vspace{.10in}%
    \begin{indented}%
       {\bf (#1)} %
}{\end{indented}}

\def\indented#1{\list{}{}\item[]}
\let\indented=\endlist
\def\show#1{\ifdefempty{#1}{}{#1\\}}
\def\IMP{\rightarrow}
\def\AND{\wedge}
\def\OR{\vee}
\def\BI{\leftrightarrow}
\def\DIFF{\setminus}
\def\SUB{\subseteq}


\newcolumntype{C}{>{\centering\arraybackslash}m{1.5cm}}
\renewcommand\qedsymbol{$\blacksquare$}
\newtcolorbox{answer}
{
  colback   = green!5!white,    % Background color
  colframe  = green!75!black,   % Outline color
  box align = center,           % Align box on text line
  varwidth upper,               % Enables multi line input
  hbox                          % Bounds box to text width
}

\newcommand{\myhwname}{Homework 1}
\newcommand{\myname}{Sebastian Liu}
\newcommand{\myemail}{ll57@cs.washington.edu}
\newcommand{\mysection}{AB}
\newcommand{\dollararrow}{\stackrel{\$}{\leftarrow}}
%%%%%%%%%%%%%%%%%%%%%%%%%%%%%%%%%%%%%%%%%%%%%%%%%%%%%%%%%%%

%%%%%%%%%%%%%%%%%%% Document Options %%%%%%%%%%%%%%%%%%%%%%
\noquestionnumbers
%%%%%%%%%%%%%%%%%%%%%%%%%%%%%%%%%%%%%%%%%%%%%%%%%%%%%%%%%%%

%%%%%%%%%%%%%%%%%%%%%%%% WORK BELOW %%%%%%%%%%%%%%%%%%%%%%%%
\begin{document}

\begin{center}
    \textbf{Homework 1} \bigskip
\end{center}

%%%%%%%%%%%%%%%%%%%%%%%% Task 1 %%%%%%%%%%%%%%%%%%%%%%%%M
\begin{question}{Task 1 - Encryption Scheme (10 points)}
    \begin{part}
       \begin{answer}
            \underline{\textbf{procedure} Dec($K = (d, \pi), C = (C[1],...,C[4])$) :} \\
            $x_0 \leftarrow d$ \\
            \textbf{for} $i = 1$ to $4$ \textbf{do} \\
            \hspace*{22pt} $x_i \leftarrow \pi^{-1}(C[i])$ \\
            \hspace*{22pt} $M[i] \leftarrow ( x_i - x_{i-1} - 1 + i )$ mod $10$ \\
            \textbf{return} $M = (M[1],...,M[4])$ 
       \end{answer}
    \end{part}

    \begin{part}
        \begin{answer}
         \textbf{Proof.} Assume $M^* = C^* = (0,0,0,0)$, \\
        \begin{align*}\underset{M \overset{\text{\$}}{\leftarrow} \mathbb{Z}_{10}^4}{\text{Pr}} [M = M^*] &= \underset{M[1] \overset{\text{\$}}{\leftarrow} \mathbb{Z}_{10}}{\text{Pr}} [M[1] = 0] \times 
            \underset{M[2] \overset{\text{\$}}{\leftarrow} \mathbb{Z}_{10}}{\text{Pr}} [M[2] = 0] \times \underset{M[3] \overset{\text{\$}}{\leftarrow} \mathbb{Z}_{10}}{\text{Pr}} [M[3] = 0] \\
            &\hspace*{20pt}\times \underset{M[4] \overset{\text{\$}}{\leftarrow} \mathbb{Z}_{10}}{\text{Pr}} [M[4] = 0]\\
            &= \frac{1}{10} \times \frac{1}{10} \times \frac{1}{10} \times \frac{1}{10} = \frac{1}{10^4}  
        \end{align*} 
        Given $C^* = (0,0,0,0)$ and $x_0 = d$, using the decryption algorithm, we can derive: \\
        \hspace*{22pt} $x_1 = x_2 = x_3 = x_4 = \pi(0)$, \\
        \hspace*{22pt} $M[1] = (\pi^{-1}(0) - d -1 + 1)$ mod $10 = (\pi^{-1}(0) - d)$ mod $10$, \\
        \hspace*{22pt} $M[2] = (\pi^{-1}(0) - \pi^{-1}(0) -1 + 2)$ mod $10 = (1)$ mod $10 = 1$, \\
        \hspace*{22pt} $M[3] = (\pi^{-1}(0) - \pi^{-1}(0) -1 + 3)$ mod $10 = (2)$ mod $10 = 2$, \\
        \hspace*{22pt} $M[4] = (\pi^{-1}(0) - \pi^{-1}(0) -1 + 4)$ mod $10 = (3)$ mod $10 = 3$, \\
        which give us: $M = (M[1],...,M[4]) = ( (\pi^{-1}(0) - d)$ mod $10, 1, 2, 3 ) \ne (0,0,0,0) = M^*$ \\
        Thus, we have: \\ $ \underset{K \overset{\text{\$}}{\leftarrow} \text{Kg}, M \overset{\text{\$}}{\leftarrow} \mathbb{Z}_{10}^4}{\text{Pr}} [M = M^* \mid \text{Enc}(K = (d, \pi), M = (M[1],...,M[4]))= C^*] = 0 \ne \frac{1}{10^4} = \underset{M \overset{\text{\$}}{\leftarrow} \mathbb{Z}_{10}^4}{\text{Pr}} [M = M^*]$
        , which is a violation of Shannon secrecy.\\\\
        Therefore, this encryption scheme is not perfectly secret.
        \end{answer}
     \end{part}
\end{question}

%%%%%%%%%%%%%%%%%%%%%%%% Task 2 %%%%%%%%%%%%%%%%%%%%%%%%
\begin{question}{Task 2 - The Shuffle (19 points)}
    \begin{part}
       \begin{answer}
            Since $\overline{M}$ is the bit-wise complement of $M$ and $M'$ is the concatenation of $M$ and $\overline{M}$,
            we know $M'$ satisfies the invariant that it has the same number of $0$'s and $1$'s. \\
            Since $\pi$ is a random permutation of $\{1,...,2n\}$ and $C[i] \leftarrow M'[\pi(i)]$, we know $C$ is effectively a random permutation of $M'$, 
            which means $C$ also satisfies the invariant that it has the same number of $0$'s and $1$'s. \\
            Additionally, since we know the length of $C$ is $2n$, $C$ must contain exactly $n$ $0$'s and $n$ $1$'s. \\
            Therefore, the ciphertext space can be described as:\\
            $\mathcal{C} = \{C \in \{0,1\}^{2n}: \text{where } C \text{ contains exactly }n \text{ 0's and } n \text{ 1's} \} $ \\
       \end{answer}
    \end{part}

    \begin{part}
        \begin{answer}
            \underline{\textbf{procedure} Dec'($\pi, C = (C[1],...,C[2n]))$ :} \\
            \textbf{for} $i = 1$ to $2n$ \textbf{do} \\
            \hspace*{22pt} $M'[\pi(i)] \leftarrow C[i]$ \\
            \textbf{for} $j = 1$ to $n$ \textbf{do} \\
            \hspace*{22pt} $M[j] \leftarrow M'[j]$ \\
            \textbf{return} $M = (M[1],...,M[n])$ 
        \end{answer}
    \end{part}

    \begin{part}
        \begin{answer}
            Since $\pi$ is a random permutation of $\{1,...,2n\}$, each bit position in the ciphertext is equally likely to be any bit
            position in $M || \overline{M}$ which contains exactly $n$ 0's and $n$ 1's (as seen in part (a)). The encryption algorithm is effectively uniformly randomly shuffling $M || \overline{M}$. \\
            Therefore, the distribution of Enc'$(\pi, M)$ is uniform over the ciphertext space $\mathcal{C}$. \\
            Assume we pick an arbitrary ciphertext $C $ from the ciphertext space $\mathcal{C}$, the distribution of Enc'$(\pi, M)$ for all ciphertext $C \in \mathcal{C}$ can be described as:
            \begin{align*}
                \underset{\pi \overset{\text{\$}}{\leftarrow}  \text{Perms}(\{1,..,2n\}), M \overset{\text{\$}}{\leftarrow} \{0,1\}^n}{\text{Pr}} [\text{Enc'}(\pi, M) = C] &= 
                \underset{C^* \overset{\text{\$}}{\leftarrow} \mathcal{C}}{\text{Pr}} [C = C^*] \\
                &= \frac{1}{|\mathcal{C}|} \\
                &= \frac{1}{\binom{2n}{n}} \\
                &= \frac{(n!)^2}{(2n)!}
            \end{align*}
        \end{answer}
    \end{part}
    \newpage
    \begin{part}
        \begin{answer}
            \textbf{Proof.} For all $M \in \{0,1\}^n$ and $C \in \mathcal{C}$, 
            \begin{align*}
                \underset{\pi \overset{\text{\$}}{\leftarrow} \text{Kg}}{\text{Pr}} [\text{Enc'}(\pi, M) = C] &= 
                \underset{\pi \overset{\text{\$}}{\leftarrow}  \text{Perms}(\{1,..,2n\})}{\text{Pr}} [\text{Enc'}(\pi, M) = C] \\
                &= \underset{C^* \overset{\text{\$}}{\leftarrow} \mathcal{C}}{\text{Pr}} [C = C^*] \tag{as explained in part(c), Enc' uniformly randomly shuffles $M || \overline{M}$}\\
                &= \frac{1}{|\mathcal{C}|} \\
                &= \frac{1}{\binom{2n}{n}} \\
                &= \frac{(n!)^2}{(2n)!}
            \end{align*}
        \end{answer}
    \end{part}

\end{question}

%%%%%%%%%%%%%%%%%%%%%%%% Task 3 %%%%%%%%%%%%%%%%%%%%%%%%
\begin{question}{Task 3 - Playing with AES (10 points)} %%May be explain more
    \begin{part}
       \begin{answer}
            AES$(X,X) =$ 24 f3 dc 26 07 11 10 ad 52 58 a4 55 67 14 d0 1d
       \end{answer}
    \end{part}

    \begin{part}
        \begin{answer}
           $C = $ AES$(X, M) =$ 00 00 00 00 00 00 00 00 00 00 00 00 00 00 00 00 when \\
           $M =$ fd e4 d4 2d 80 2d 9e 09 18 fd 5f ae 0c 6f a2 9c  \\
           I found $M$ by running the InvertAES function with $K = X$ and $C = $ 00 00 00 00 00 00 00 00 00 00 00 00 00 00 00 00.\\
           It works because block cipher creates a permutation of the set of all possible 128-bit blocks, so there must be a $M$ that satisfy 
           the requirement of $C$ with the given key $X$.
        \end{answer}
     \end{part}

     \begin{part}
        \begin{answer}
            $C = $ AES$(K, X) =$ 37 d9 12 89 07 fa 24 b0 17 b1 04 b2 aa ee 5e 00 when \\
            $K =$ 1c 0b bc 7f 17 0d bf d6 8d c6 8d 37 d5 6c 71 cf \\
            I found $K$ by brute forcing the key space (i.e. I wrote a while loop which generate a random 16-byte number as $K$ for each iteration 
            and return the first $K$ whose AES$(K, X)$ output has its last byte as $00$).
        \end{answer}
     \end{part}

\end{question}

%%%%%%%%%%%%%%%%%%%%%%%% Task 4 %%%%%%%%%%%%%%%%%%%%%%%%
\begin{question}{Task 4 - Distinguishing Advantage (6 points)}
    \begin{part}
       \begin{answer}
        \textbf{Fact 1:} Pr$[D_1^{O_0} \Rightarrow 1] =  \underset{b_1 \overset{\text{\$}}{\leftarrow} \{0,1\}}{\text{Pr}}[b_1 = 1] = \frac{1}{2}$ \\\\
        \textbf{Fact 2:} Pr$[D_1^{O_1} \Rightarrow 1] = \underset{b_1 \overset{\text{\$}}{\leftarrow} \{0,1\}}{\text{Pr}}[b_1 = 1] = \frac{1}{2}$ \\\\
        $\textbf{Adv}_{O_0,O_1}^{\text{dist}}(D_1) = \lvert \frac{1}{2} - \frac{1}{2} \rvert = 0$
       \end{answer}
    \end{part}

    \begin{part}
        \begin{answer}
            \textbf{Fact 1:} Pr$[D_2^{O_0} \Rightarrow 1] = \underset{b_1 \overset{\text{\$}}{\leftarrow} \{0,1\}, b_2 \overset{\text{\$}}{\leftarrow} \{0,1\}}{\text{Pr}}[b_1 \oplus b_2 = 1] \\\\
            \hspace*{100pt}= \underset{b_1 \overset{\text{\$}}{\leftarrow} \{0,1\}}{\text{Pr}}[b_1 = 1] \times  \underset{b_2 \overset{\text{\$}}{\leftarrow} \{0,1\}}{\text{Pr}}[b_2 = 0] + 
            \underset{b_1 \overset{\text{\$}}{\leftarrow} \{0,1\}}{\text{Pr}}[b_1 = 0] \times  \underset{b_2 \overset{\text{\$}}{\leftarrow} \{0,1\}}{\text{Pr}}[b_2 = 1]\\\\
            \hspace*{100pt}= \frac{1}{2} \times \frac{1}{2} + \frac{1}{2} \times \frac{1}{2} \\\\
            \hspace*{100pt}= \frac{1}{2}$ \\\\

            \textbf{Fact 2:} Pr$[D_2^{O_1} \Rightarrow 1] = \underset{b_1 \overset{\text{\$}}{\leftarrow} \{0,1\}, b_2 \overset{b_1}{\leftarrow} \{0,1\}}{\text{Pr}}[b_1 \oplus b_2 = 1] \\\\
            \hspace*{100pt}= \underset{b_1 \overset{\text{\$}}{\leftarrow} \{0,1\}}{\text{Pr}}[b_1 = 1] \times  \underset{b_2 \leftarrow 0}{\text{Pr}}[b_2 = 0] + 
            \underset{b_1 \overset{\text{\$}}{\leftarrow} \{0,1\}}{\text{Pr}}[b_1 = 0] \times  \underset{b_2 \overset{\text{\$}}{\leftarrow} \{0,1\}}{\text{Pr}}[b_2 = 1]\\\\
            \hspace*{100pt}= \frac{1}{2} \times 1 + \frac{1}{2} \times \frac{1}{2} \\\\
            \hspace*{100pt}= \frac{3}{4}$ \\\\

            $\textbf{Adv}_{O_0,O_1}^{\text{dist}}(D_2) = \lvert \frac{3}{4} - \frac{1}{2} \rvert = \frac{1}{4}$
        \end{answer}
     \end{part}

\end{question}
\end{document}