%%%%%%%%%%%%%%%%%%%%% PACKAGE IMPORTS %%%%%%%%%%%%%%%%%%%%%
\documentclass[11pt]{article}
\usepackage{amsmath, amsfonts, amsthm, amssymb}
\usepackage{lmodern}
\usepackage{microtype}
\usepackage{fullpage}       
\usepackage{changepage}
\usepackage{hyperref}
\usepackage{blindtext}
\hypersetup{
    colorlinks=true,
    linkcolor=blue,
    filecolor=magenta,      
    urlcolor=blue,
    pdftitle={Overleaf Example},
    pdfpagemode=FullScreen,
    }
\urlstyle{same}

\newenvironment{level}%
{\addtolength{\itemindent}{2em}}%
{\addtolength{\itemindent}{-2em}}

\usepackage{amsmath,amsthm,amssymb}


\usepackage[x11names, rgb]{xcolor}
\usepackage{graphicx}
\usepackage[nooldvoltagedirection]{circuitikz}
\usetikzlibrary{decorations,arrows,shapes}

\usepackage{datetime}
\usepackage{etoolbox}
\usepackage{enumerate}
\usepackage{enumitem}
\usepackage{listings}
\usepackage{array}
\usepackage{varwidth}
\usepackage{tcolorbox}
\usepackage{amsmath}
\usepackage{circuitikz}
\usepackage{verbatim}
\usepackage[linguistics]{forest}
\usepackage{listings}
\usepackage{xcolor}
\renewcommand{\rmdefault}{cmss}


\newcommand\doubleplus{+\kern-1.3ex+\kern0.8ex}
\newcommand\mdoubleplus{\ensuremath{\mathbin{+\mkern-10mu+}}}

\definecolor{codegreen}{rgb}{0,0.6,0}
\definecolor{codegray}{rgb}{0.5,0.5,0.5}
\definecolor{codepurple}{rgb}{0.58,0,0.82}
\definecolor{backcolour}{rgb}{0.95,0.95,0.92}

\lstdefinelanguage{JavaScript}{
  keywords={typeof, new, true, false, catch, function, return, null, catch, switch, var, if, in, while, do, else, case, break},
  keywordstyle=\color{blue}\bfseries,
  ndkeywords={class, export, boolean, throw, implements, import, this},
  ndkeywordstyle=\color{darkgray}\bfseries,
  identifierstyle=\color{black},
  sensitive=false,
  comment=[l]{//},
  morecomment=[s]{/*}{*/},
  commentstyle=\color{purple}\ttfamily,
  stringstyle=\color{red}\ttfamily,
  morestring=[b]',
  morestring=[b]"
}

\lstdefinestyle{mystyle}{
    language=JavaScript,
    backgroundcolor=\color{backcolour},   
    commentstyle=\color{codegreen},
    keywordstyle=\color{magenta},
    numberstyle=\tiny\color{codegray},
    stringstyle=\color{codepurple},
    basicstyle=\ttfamily\footnotesize,
    breakatwhitespace=false,         
    breaklines=true,                 
    captionpos=b,                    
    keepspaces=true,                 
    numbers=left,                    
    numbersep=5pt,                  
    showspaces=false,                
    showstringspaces=false,
    showtabs=false,                  
    tabsize=2
}

\lstset{style=mystyle}
\setlength{\parindent}{0pt}
\setlength{\parskip}{5pt plus 1pt}

\providetoggle{questionnumbers}
\settoggle{questionnumbers}{true}
\newcommand{\noquestionnumbers}{
    \settoggle{questionnumbers}{false}
}

\newcounter{questionCounter}
\newenvironment{question}[2][\arabic{questionCounter}]{%
    \ifnum\value{questionCounter}=0 \else {\newpage}\fi%
    \setcounter{partCounter}{0}%
    \vspace{.25in} \hrule \vspace{0.5em}%
    \noindent{\bf \iftoggle{questionnumbers}{Question #1: }{}#2}%
    \addtocounter{questionCounter}{1}%
    \vspace{0.8em} \hrule \vspace{.10in}%
}

\newcounter{partCounter}[questionCounter]
\renewenvironment{part}[1][\alph{partCounter}]{%
    \addtocounter{partCounter}{1}%
    \vspace{.10in}%
    \begin{indented}%
       {\bf (#1)} %
}{\end{indented}}

\def\indented#1{\list{}{}\item[]}
\let\indented=\endlist
\def\show#1{\ifdefempty{#1}{}{#1\\}}
\def\IMP{\rightarrow}
\def\AND{\wedge}
\def\OR{\vee}
\def\BI{\leftrightarrow}
\def\DIFF{\setminus}
\def\SUB{\subseteq}


\newcolumntype{C}{>{\centering\arraybackslash}m{1.5cm}}
\renewcommand\qedsymbol{$\blacksquare$}
\newtcolorbox{answer}
{
  colback   = green!5!white,    % Background colorucyitc,
  colframe  = green!75!black,   % Outline color
  box align = center,           % Align box on text line
  varwidth upper,               % Enables multi line input
  hbox                          % Bounds box to text width
}

\newcommand{\myhwname}{Homework 4}
\newcommand{\myname}{Sebastian Liu}
\newcommand{\myemail}{ll57@cs.washington.edu}
\newcommand{\mysection}{AB}
\newcommand{\dollararrow}{\stackrel{\$}{\leftarrow}}
%%%%%%%%%%%%%%%%%%%%%%%%%%%%%%%%%%%%%%%%%%%%%%%%%%%%%%%%%%%

%%%%%%%%%%%%%%%%%%% Document Options %%%%%%%%%%%%%%%%%%%%%%
\noquestionnumbers
%%%%%%%%%%%%%%%%%%%%%%%%%%%%%%%%%%%%%%%%%%%%%%%%%%%%%%%%%%%

%%%%%%%%%%%%%%%%%%%%%%%% WORK BELOW %%%%%%%%%%%%%%%%%%%%%%%%
\begin{document}

\begin{center}
    \textbf{Homework 4} \bigskip
\end{center}

%%%%%%%%%%%%%%%%%%%%%%%% Task 1 %%%%%%%%%%%%%%%%%%%%%%%%M
\begin{question}{Task 1 - Encrypt \& MAC (15 points)}
    \begin{part}
       \begin{answer}
        \textbf{From the ciphertexts, we can infer that the plaintexts are the same.}\\\\
        \textbf{Justification:}\\
        We can see that the last 256 bits of both ciphertexts $C_1$ and $C_2$ are the same. 
        Given that the tag produced by the MAC algorithm is also 256 bits, we can infer that
        these common bits are the MAC tags. \\\\
        Since the MACs are done on the plaintexts and both ciphertexts are generated using the same secret key, 
        given that the MAC tags are equal, this implies that the actual plaintexts are the same.\\\\
        Additionally, since both ciphertexts are encrypted using CTR with AES, the encryption algorithm will produce different ciphertexts
        even if the plaintexts are the same (assuming $C_1$ and $C_2$ used different initialization vector). Thus, the fact that the first 
        50 bytes of $C_1$ and $C_2$ are different doesn't contradict our inference that the plaintexts are the same.
       
        \end{answer}
    \end{part}

    \begin{part}
        \begin{answer}
         \textbf{E\&M is a good scheme in terms of IND-CPA security.}\\\\
         \textbf{Explanation:}\\
         Since HMAC with SHA-256 produces tags which are computationally indistinguishable from random by someone without the key, the tags themselves do not reveal any
         information about the plaintext. Since CTR mode with AES is IND-CPA secure, the ciphertext produced does not allow an attacker to determine which plaintext was
         encrypted.\\\\
        Appending a MAC tag to the ciphertext is similar to appending a fixed pattern to an IND-CPA secure ciphertext, which we have proven does not impact
        the scheme's IND-CPA security in Homework 2 Task 4b). The tag acts similar to a fixed suffix that provides integrity without compromising confidentiality 
        from the CTR encryption.\\\\
        Therefore, the E\&M is a good scheme in terms of IND-CPA security.
         \end{answer}
     \end{part}
\newpage
     \begin{part}
        \begin{answer}
            The main observation is that the E\&M scheme produces identical tags for the same plaintexts since SHA-256 is deterministic.
            Let $\Pi = (\text{Kg}, \text{Enc}, \text{Dec})$
            be the encryption scheme E\&M, We can construct an adversary against $\Pi$ in the following way:
            \begin{center}
            \begin{tabular}{l}
                    \underline{\textbf{adversary} $A^{O}()$:} \\
                    $C'_1[0]C'_1[1]...C'_1[\ell]\;||\;T_1 \leftarrow O.\text{Encrypt}(0^{n \cdot \ell})$ \\
                    $C'_2[0]C'_2[1]...C'_2[2\ell]\;||\;T_2 \leftarrow O.\text{Encrypt}(0^{n \cdot 2\ell})$ \\
                    $C^* = C'_2[0]C'_2[1]...C'_2[\ell] \;||\;T_1$ \\
                    \textbf{return} $C^*$
                \end{tabular}
            \end{center}
                Consider the INT-CTXT oracle. When running $\text{A}^{\text{INT-CTXT}[E\&M]}$, for the adversary to succeed, 
                two things need to be satisfied:
                $(1)\; C^* \notin \mathcal{Q}$, and $(2)\;\text{Dec}(K, C^*) \ne \bot$. \\
                
                Firstly, the oracle's state is \(\mathcal{Q} = \{C_1, C_2\}\), where \(C_1 = C'_1[0]C'_1[1]...C'_1[\ell]\;||\;T_1\) 
                from \(\text{Enc}(K, 0^{n\cdot \ell})\), and \(C_2 = C'_2[0]C'_2[1]...C'_2[2\ell]\;||\;T_2\) from
                 \(\text{Enc}(K, 0^{n \cdot 2\ell})\). Assuming the IV is different for each encryption, \(C'_2[0] \ne C'_1[0]\) 
                 and because \(C^*\) is composed only half of the blocks from \(C_2\) with the tag \(T_1\) from \(C_1\), we have that 
                 \(C_1 \ne C_2 \ne C^*\), and thus, \(C^* \notin \mathcal{Q}\).\\\\
                Secondly, by the E\&M construction, decrypting the counter-mode ciphertexts of \(C_1\) and \(C^*\) will yield the same
                plaintext, because the first \(\ell\) blocks of \(0^{n \cdot 2\ell}\) are the same as \(0^{n \cdot \ell}\), and CTR 
                blocks can be decrypted independently. Since SHA-256 is deterministic, running the HMAC algorithm on the decrypted 
                counter-mode ciphertext of \(C^*\) will produce the same tag as \(T_1\). Therefore, 
                \(\text{Dec}(K, C^*)\) returns \(M \ne \bot\).\\
                In conclusion,
                \begin{align*}
                    \text{Adv}_{\Pi}^{\text{int-ctxt}}(A) &= \Pr[\text{A}^{\text{INT-CTXT}[\Pi]} \Rightarrow 1] \\
                    &= \Pr[ C^* \notin \mathcal{Q} \wedge \text{Dec}(K, C^*) \ne \bot] = 1,
                \end{align*}
                and since $A$ returns in polynomial time, this means $\Pi$ is not INT-CTXT secure.
                
        \end{answer}
     \end{part}

\end{question}

%%%%%%%%%%%%%%%%%%%%%%%% Task 2 %%%%%%%%%%%%%%%%%%%%%%%%
\begin{question}{Task 2 - Authenticated Encryption (15 points)}
    \begin{part}
       \begin{answer}
         \textbf{The scheme doesn't guarantee ciphertext integrity.} \\
         \textbf{Explanation:}\\
            The main observation is that the encryption scheme ensures the integrity of the decrypted plaintext by XORing
            all plaintext blocks. Since counter-mode encryption is malleable, two bit-flips at the same index in different
            blocks will cancel each other out. This allows us to construct ciphertexts that will pass the integrity check
            but differ from the original ciphertext. Let $\Pi = (\text{Kg}, \text{Enc}, \text{Dec})$ be the AES-based CTR encryption
                scheme described in the question. We can construct an adversary against $\Pi$ as follows:
            \begin{center}
             \begin{tabular}{l}
                    \underline{\textbf{adversary} $A^{O}()$:} \\
                    $C[0]||C[1]||C[2]||C[3]||C[4] \leftarrow O.\text{Encrypt}(0^{256})$ \\
                    $C^*[1] \leftarrow C[1] \text{ with the first bit flipped}$\\
                    $C^*[2] \leftarrow C[2] \text{ with the first bit flipped}$ \\
                    
                    $C^* = C[0] || C^*[1] || C^*[2] || C[3] || C[4]$ \\
                    \textbf{return} $C^*$
                \end{tabular}
            \end{center}
            Consider the INT-CTXT oracle. When running $\text{A}^{\text{INT-CTXT}[\Pi]}$, for the adversary to succeed, two 
            conditions must be met: (1) $C^* \notin \mathcal{Q}$, and (2) $\text{Dec}(K, C^*) \neq \bot$. \\\\
            Firstly, the oracle's state is $\mathcal{Q} = \{C\}$, where $C = C[0]||C[1]||C[2]||C[3]||C[4]$ from 
            $\text{Enc}(K, 0^{256})$. Given the 256-bit plaintext and 128-bit block size, we know that $C[0]$ is the 
            initialization vector, $C[1]$ and $C[2]$ are the encrypted plaintext blocks, $C[3]$ is the padding block, 
            and $C[4]$ is the additional integrity checking block. As $C^*$ has the first bits of $C[1]$ and $C[2]$ flipped, 
            it is clear that $C \neq C^*$, and thus, $C^* \notin \mathcal{Q}$.\\\\\
            Secondly, by the construction of the encryption scheme, $\text{Dec}(K, C^*)$ recovers $M^*[1], M^*[2], M[3], M[4]$ 
            from $C^*$ in counter mode. Due to the malleability of counter mode, the first bit of $M^*[1]$ and $M^*[2]$ will 
            be flipped compared to the original plaintext blocks $M[1]$ and $M[2]$. However, since the flips occur at the 
            same position, $M^*[1] \oplus M^*[2] \oplus M[3] = M[1] \oplus M[2] \oplus M[3] = M[4]$. Since the padding 
            block $M[3]$ is unchanged, $M^*[1], M^*[2], M[3]$ will pass the padding validation. Therefore, $\text{Dec}(K, C^*)$ 
            returns $M \neq \bot$.
            
            In conclusion, the adversary's advantage is computed as
            \begin{align*}
                \text{Adv}_{\Pi}^{\text{int-ctxt}}(A) &= \Pr[\text{A}^{\text{INT-CTXT}[\Pi]} \Rightarrow 1] \\
                &= \Pr[ C^* \notin \mathcal{Q} \wedge \text{Dec}(K, C^*) \neq \bot] = 1,
            \end{align*}
            which indicates that $\Pi$ is not INT-CTXT secure since $A$ operates in polynomial time.
        \end{answer}
    \end{part}
\newpage
    \begin{part}
        \begin{answer}
            \textbf{The modified scheme still doesn't satisfy ciphertext integrity.} \\
            \textbf{Explanation:}\\
            The main observation is that the modified scheme remains vulnerable 
            to malleability under counter mode encryption. Flipping the same bit in two 
            different blocks of the plaintext will not change the result of the XOR operation. Since AES is A
            deterministic function, an adversary can still produce a modified ciphertext that, when decrypted, 
            still passes the integrity check.

            Consider the modified AES-based CTR encryption scheme $\Pi' = (\text{Kg'}, \text{Enc'}, \text{Dec'})$ with an 
            added integrity check, where the last ciphertext block is computed as 
            $C[\ell + 1] = \text{AES}_K (\text{IV} \oplus M[1] \oplus \cdots \oplus M[\ell])$. 
            We can reuse the adversary from part (a) against the integrity of $\Pi'$ as follows:
            \begin{center}
            \begin{tabular}{l}
                \underline{\textbf{adversary} $A^{O'}()$:} \\
                $C[0]||C[1]||C[2]||C[3]||C[4] \leftarrow O'.\text{Encrypt}(0^{256})$ \\
                $C^*[1] \leftarrow C[1] \text{ with the first bit flipped}$\\
                $C^*[2] \leftarrow C[2] \text{ with the first bit flipped}$ \\
                
                $C^* = C[0] || C^*[1] || C^*[2] || C[3] || C[4]$ \\
                \textbf{return} $C^*$
            \end{tabular}
            \end{center}

            Consider the INT-CTXT security definition. For adversary $A$ to succeed against the integrity of $\Pi'$, 
            two conditions must be satisfied: $(1)\; C^* \notin \mathcal{Q}$, and $(2)\; \text{Dec'}(K, C^*) \ne \bot$. \\\\

            Firstly, same as in part(a), since $C^*$ has the first bits of $C[1]$ and $C[2]$ flipped, we have \(C \ne C^*\), and thus, 
            \(C^* \notin \mathcal{Q}\).\\\\

            Secondly, $\text{Dec'}(K, C^*)$ first recovers $M^*[1],M^*[2],M[3]$ from $C^*$. 
            Due to the malleability of counter mode, the first bit of $M^*[1]$ and $M^*[2]$ are flipped 
            from the original plaintext blocks $M[1]$ and $M[2]$. However, the bit flips in $M^*[1]$ and $M^*[2]$
             cancel out in the XOR operation used for integrity check. Thus,
              $$\text{AES}_K (\text{IV} \oplus M^*[1] \oplus M^*[2] \oplus M[3]) = \text{AES}_K (\text{IV} \oplus M[1] \oplus M[2] \oplus M[3]) = C[4]$$ Since the padding block
               $M[3]$ remains unchanged, $M^*[1],M^*[2],M[3]$ will pass the padding and integrity checks. Therefore,
                $\text{Dec'}(K, C^*)$ returns $M \ne \bot$.

            In conclusion, the advantage of the adversary $A$ in breaking the integrity of $\Pi'$ is:
            \begin{align*}
                \text{Adv}_{\Pi'}^{\text{int-ctxt}}(A) &= \Pr[\text{A}^{\text{INT-CTXT}[\Pi']} \Rightarrow 1] \\
                &= \Pr[ C^* \notin \mathcal{Q} \wedge \text{Dec'}(K, C^*) \ne \bot] = 1,
            \end{align*}
            and since $A$ operates in polynomial time, this implies that $\Pi'$ is not INT-CTXT secure under this modified scheme.

        \end{answer}
    \end{part}

\end{question}

%%%%%%%%%%%%%%%%%%%%%%%% Task 3 %%%%%%%%%%%%%%%%%%%%%%%%
\begin{question}{Task 3 - Set Commitments (10 points)} 
    \begin{part}
       \begin{answer}
        To verify the integrity of the file \( F_i \) with respect to \( \text{com} \), the server needs to provide the client with:
        (1) The hash value of \( F_i \), \( H(S, F_i) \), 
        (2) The sibling hash values for each level of the tree on the path from \( F_i \) to the root.\\
        In the example below with \( m = 8 \), the client wants to retrieve and verify the integrity of \( F_4 \). The server sends 
        client the file \( F_4 \), the $\text{com}$ (the root), and the sibling hash values for each level of the tree (marked in red, [$h_3, h_a,$ and $h_y$]) on 
        the path from \( F_4 \) to the root (marked in green). \\
        \includegraphics*[width=0.8\linewidth]{merkle.png}\\
        The client can verify the integrity of \( F_i\) by doing the following:\\
        1. Compute \( H(S, F_i) \) to get the hash of the file.\\
        2. At each level of the binary tree, the client computes the parent node hash by concatenating the hash of \( F_i \) and its
         sibling with the sibling's hash provided by the server, and then hashing the result with the seed \( S \). \\
        3. The client repeats this process up the tree, each time use the newly computed hash along with the next sibling hash 
        provided by the server.\\
        4. When the client reaches the root, they compare the computed root hash with the commitment \( \text{com} \) published 
        by the server. If they match, then file \( F_i \) is valid.\\\\
        The content sent by the server (excluding \( F_i \) itself) is \( O(n \log m) \) bits long because each hash is \( n \) bits long and there are 
        \( \log m \) sibling hashes to send (for a balanced binary tree with $m$ leaves, the path from any leaf to the root has $\log m$ levels [since $m$ is a power of two], and we have
        one sibling on each level).\\
        Because the hash function \( H \) is collision-resistant, it is computationally infeasible for an adversary to find
        two different inputs that hash to the same output. So changing any file \( F_i \) would result in a different hash 
        at the leaf, and change would propagate up the tree result in a different root hash. Therefore, any change in the files would 
        be detected when the client computes a root hash that does not match the commitment.
       \end{answer}
    \end{part}

\end{question}

%%%%%%%%%%%%%%%%%%%%%%%% Task 4 %%%%%%%%%%%%%%%%%%%%%%%%
\begin{question}{Task 4 - Number Theory \& Groups (10 points)} 
    \begin{part}
       \begin{answer}
            By the definition, $\mathbb{Z}_{35}^* = \{a \in \mathbb{Z}_{35}: \mathsf{gcd}(a,35)=1\}$. Essentially, we are looking
            for all numbers between 0 and 34 that are coprime with 35. Since 35 is the product of two prime numbers 5 and 7, any number
            that is not a multiple of 5 or 7 is coprime with 35. Therefore, $\mathbb{Z}_{35}^* = \{1,2,3,4,6,8,9,11,12,13,16,17,18,19,22,23,24,26,27,29,31,32,33,34\}$.

       \end{answer}
    \end{part}

    \begin{part}
        \begin{answer}
            Since 37 is a prime number, $\mathbb{Z}_{37}^* = \{a: 1 \leq a \leq 36\}$. When \textcolor{red}{$g = 2$}, we have the following: \\\\
            \begin{tabular}{|c|c|c|c|c|c|c|c|c|c|c|c|c|c|c|c|c|c|c|c|c|c|c|c|c|c|c|c|c|c|c|}
                \hline
                e = & 0 & 1 & 2 & 3 & 4 &  5 & 6 & 7 & 8 & 9 & 10 & 11 & 12 & 13 & 14 & 15 & 16 & 17\\
                \hline
                     & \textcolor{red}{1} & \textcolor{red}{2} & \textcolor{red}{4} & \textcolor{red}{8} & \textcolor{red}{16} & \textcolor{red}{32} & \textcolor{red}{27} & \textcolor{red}{17} & \textcolor{red}{34} & \textcolor{red}{31} & \textcolor{red}{25} & \textcolor{red}{13} & \textcolor{red}{26} & \textcolor{red}{15} & \textcolor{red}{30} & \textcolor{red}{23} & \textcolor{red}{9} & \textcolor{red}{18}\\
                 \hline
                e =  & 18 & 19 & 20 & 21 & 22 & 23 & 24 & 25 & 26 & 27 & 28 & 29 &  30 & 31 & 32 & 33 & 34 & 35 \\
                \hline
                & \textcolor{red}{36} & \textcolor{red}{35} & \textcolor{red}{33} & \textcolor{red}{29} & \textcolor{red}{21} & \textcolor{red}{5} & \textcolor{red}{10} & \textcolor{red}{20} & \textcolor{red}{3} & \textcolor{red}{6} & \textcolor{red}{12} & \textcolor{red}{24} & \textcolor{red}{11} & \textcolor{red}{22} & \textcolor{red}{7} & \textcolor{red}{14} & \textcolor{red}{28} & \textcolor{red}{19} \\
                \hline
                \end{tabular}
                \\\\
                So, we have: $\langle 2 \rangle = \{2^0 = 1, 2^1,...,2^{36 - 1}\} = \mathbb{Z}_{37}^*$. \\
                Therefore, $2$ is a generator of $\mathbb{Z}_{37}^*$.
        \end{answer}
    \end{part}

    \begin{part}
        \begin{answer}
            To find $x \in \mathbb{Z}_{187}$ such that $125 x \equiv 4 \; (\mathsf{mod} \; 187)$, we are essentially looking for $x,y \in \mathbb{Z}_{187}$ such 
            that $125x + 187y = 4$. We can use the extended Euclidean algorithm to find $x$ and $y$: \\
            \begin{tabular}{|c|c|}
                \hline
                Euclidean algorithm & Backtrack\\
                \hline
               $ 187 = \textcolor{teal}{125} \cdot 1 + \textcolor{red}{62}$ & $ \textcolor{teal}{125} - \textcolor{red}{62} \cdot 2 = 1$\\
                \hline
                $ \textcolor{teal}{125} = \textcolor{red}{62} \cdot 2 + 1$ & $\textcolor{teal}{125} - (\textcolor{red}{187 - 125}) \cdot 2 = 1$\\
                \hline
            \end{tabular}\\\\
            Here we get $125 - (187 - 125) \cdot 2 = 125 \cdot 3 + 187 \cdot (-2) = 1$. Multiplying both sides by 4, we get $125 \cdot 12 + 187 \cdot (-8) = 4$.
            Thus, we have $x = 12$ and $y = -8$.\\
            Therefore, $x = 12$ is the solution to $125x \equiv 4 \; (\mathsf{mod} \; 187)$.
        \end{answer}
    \end{part}

\end{question}
\end{document}