%%%%%%%%%%%%%%%%%%%%% PACKAGE IMPORTS %%%%%%%%%%%%%%%%%%%%%
\documentclass[11pt]{article}
\usepackage{amsmath, amsfonts, amsthm, amssymb}
\usepackage{lmodern}
\usepackage{microtype}
\usepackage{fullpage}       
\usepackage{changepage}
\usepackage{hyperref}
\usepackage{blindtext}
\hypersetup{
    colorlinks=true,
    linkcolor=blue,
    filecolor=magenta,      
    urlcolor=blue,
    pdftitle={Overleaf Example},
    pdfpagemode=FullScreen,
    }
\urlstyle{same}

\newenvironment{level}%
{\addtolength{\itemindent}{2em}}%
{\addtolength{\itemindent}{-2em}}

\usepackage{amsmath,amsthm,amssymb}


\usepackage[x11names, rgb]{xcolor}
\usepackage{graphicx}
\usepackage[nooldvoltagedirection]{circuitikz}
\usetikzlibrary{decorations,arrows,shapes}

\usepackage{datetime}
\usepackage{etoolbox}
\usepackage{enumerate}
\usepackage{enumitem}
\usepackage{listings}
\usepackage{array}
\usepackage{varwidth}
\usepackage{tcolorbox}
\usepackage{amsmath}
\usepackage{circuitikz}
\usepackage{verbatim}
\usepackage[linguistics]{forest}
\usepackage{listings}
\usepackage{xcolor}
\renewcommand{\rmdefault}{cmss}


\newcommand\doubleplus{+\kern-1.3ex+\kern0.8ex}
\newcommand\mdoubleplus{\ensuremath{\mathbin{+\mkern-10mu+}}}

\definecolor{codegreen}{rgb}{0,0.6,0}
\definecolor{codegray}{rgb}{0.5,0.5,0.5}
\definecolor{codepurple}{rgb}{0.58,0,0.82}
\definecolor{backcolour}{rgb}{0.95,0.95,0.92}

\lstdefinestyle{mystyle}{
    language=Python,
    basicstyle=\ttfamily\small,
    keywordstyle=\color{blue},
    stringstyle=\color{red},
    commentstyle=\color{green},
    morecomment=[l][\color{magenta}]{\#},
    backgroundcolor=\color{backcolour},   
    breakatwhitespace=false,         
    breaklines=true,                 
    captionpos=b,                    
    keepspaces=true,                 
    numbers=left,                    
    numbersep=5pt,                  
    showspaces=false,                
    showstringspaces=false,
    showtabs=false,                  
    tabsize=2
}

\lstset{style=mystyle}
\setlength{\parindent}{0pt}
\setlength{\parskip}{5pt plus 1pt}

\providetoggle{questionnumbers}
\settoggle{questionnumbers}{true}
\newcommand{\noquestionnumbers}{
    \settoggle{questionnumbers}{false}
}

\newcounter{questionCounter}
\newenvironment{question}[2][\arabic{questionCounter}]{%
    \ifnum\value{questionCounter}=0 \else {\newpage}\fi%
    \setcounter{partCounter}{0}%
    \vspace{.25in} \hrule \vspace{0.5em}%
    \noindent{\bf \iftoggle{questionnumbers}{Question #1: }{}#2}%
    \addtocounter{questionCounter}{1}%
    \vspace{0.8em} \hrule \vspace{.10in}%
}

\newcounter{partCounter}[questionCounter]
\renewenvironment{part}[1][\alph{partCounter}]{%
    \addtocounter{partCounter}{1}%
    \vspace{.10in}%
    \begin{indented}%
       {\bf (#1)} %
}{\end{indented}}

\def\indented#1{\list{}{}\item[]}
\let\indented=\endlist
\def\show#1{\ifdefempty{#1}{}{#1\\}}
\def\IMP{\rightarrow}
\def\AND{\wedge}
\def\OR{\vee}
\def\BI{\leftrightarrow}
\def\DIFF{\setminus}
\def\SUB{\subseteq}


\newcolumntype{C}{>{\centering\arraybackslash}m{1.5cm}}
\renewcommand\qedsymbol{$\blacksquare$}
\newtcolorbox{answer}
{
  colback   = green!5!white,    % Background colorucyitc,
  colframe  = green!75!black,   % Outline color
  box align = center,           % Align box on text line
  varwidth upper,               % Enables multi line input
  hbox                          % Bounds box to text width
}

\newcommand{\myhwname}{Homework 6}
\newcommand{\myname}{Sebastian Liu}
\newcommand{\myemail}{ll57@cs.washington.edu}
\newcommand{\mysection}{AB}
\newcommand{\dollararrow}{\stackrel{\$}{\leftarrow}}
%%%%%%%%%%%%%%%%%%%%%%%%%%%%%%%%%%%%%%%%%%%%%%%%%%%%%%%%%%%

%%%%%%%%%%%%%%%%%%% Document Options %%%%%%%%%%%%%%%%%%%%%%
\noquestionnumbers
%%%%%%%%%%%%%%%%%%%%%%%%%%%%%%%%%%%%%%%%%%%%%%%%%%%%%%%%%%%

%%%%%%%%%%%%%%%%%%%%%%%% WORK BELOW %%%%%%%%%%%%%%%%%%%%%%%%
\begin{document}

\begin{center}
    \textbf{Homework 6} \bigskip
\end{center}

%%%%%%%%%%%%%%%%%%%%%%%% Task 1 %%%%%%%%%%%%%%%%%%%%%%%%M
\begin{question}{Task 1 - RSA Modulus Generation (10 points)}
    \begin{part}
       \begin{answer}
            Given that $N = PQ$, $P = R - i$ and $Q = R + j$, where $R$ is a k-bit integer and (since primes are dense) $i, j$ are relatively small integers, we have two scenarios:\\
            1. If $R$ is not a prime, then $P$ and $Q$ are neighboring primes in the set of $k$-bit numbers.\\
            2. If $R$ is a prime, then $P$ and $Q$ are immediate prime neighbors of $R$.\\\\
            In both scenarios, since $P$ and $Q$ are close to each other, the value of $N$ can be efficiently factorized.
            One approach can be to search for primes near the square root of $N$, which would be approximately $\sqrt{N} \approx R$. 
            This search is feasible in polynomial time relative to $k$. Once an attacker can factor \( N \) and find \( P \) and \( Q \), 
            they can efficiently compute the private key by using the public key $PK(N, e)$ and looking for a decryption exponent
            $d$ in $\mathbb{Z}_{(P-1)(Q-1)}^*$ , such that $e d = 1 \text{ (mod } (P-1)(Q -1))$.\\\\
            The proximity of $P$ and $Q$ ( $|P - Q|$  is small) significantly reduces the complexity of the factorization problem. 
            Since RSA security relies on the difficulty of factoring $N$, the method mentioned above of generating
            $P$ and $Q$ compromises the security of RSA modulus.
           
        \end{answer}
    \end{part}
\newpage
    \begin{part}
        \begin{answer}
             $P = 35123014591230139123011933120312223198716238123918231119382061$ \\
             $Q = 35123014591230139123011933120312223198716238123918231119382447$ 
         \end{answer}
         \begin{lstlisting}
    from sympy import isprime
    from timeit import default_timer as timer
    import math

    # The RSA modulus N
    N = 12336261539757652568320691057196254494530050076556470009232333
        67120767290238588667397052161653352801437540471197470570083267

    def factor(N):
        # Approximate square root of N
        approx_sqrt_N = int(math.isqrt(N))

        # Search for prime factors near the square root of N
        for i in range(approx_sqrt_N, 1, -1):
            print("Trying i = ", i)
            if N % i == 0 and isprime(i):
                # double check
                if isprime(N // i) and i * (N // i) == N:
                    return i, N // i

        print("Error: no factors found")
        return None, None

    # Find P and Q
    timer_start = timer()
    P, Q = factor(N)
    timer_end = timer()
    print("P = ", P, "Q = ", Q, "Time = ", timer_end - timer_start)
        \end{lstlisting}
     \end{part}

\end{question}

%%%%%%%%%%%%%%%%%%%%%%%% Task 2 %%%%%%%%%%%%%%%%%%%%%%%%
\begin{question}{Task 2 - ElGamal and DDH (15 points)}
    \begin{part}
        \begin{answer}
            \textbf{Decryption Algorithm:}
            \begin{center}
                \begin{tabular}{|l|}
                    \hline
                    \underline{\textbf{procedure} Dec($\text{SK}, C = (C_1, C_2)$) :} \\
                    \textbf{if} $C_1^{\text{SK}} = C_2$ \textbf{then} \\
                    \hspace*{22pt} \textbf{return} $0$ \\
                    \textbf{else} \\
                    \hspace*{22pt} \textbf{return} $1$\\
                    \hline
                \end{tabular}
            \end{center}
            \textbf{Correctness when $b = 0$:} \\
            - During encryption, we have \( C_1 = g^y \) and \( C_2 = PK^y = (g^x)^y = g^{xy} \).\\
            - During decryption, we have \( C_1^{\text{SK}} = (g^y)^x = g^{yx} = g^{xy} = C_2 \).\\
            - Thus, when $b = 0$, the decryption algorithm will always correctly output $0$.\\\\
            \textbf{Correctness when $b = 1$:} \\
            - During encryption, we have \( C_1 = g^y \) and \( C_2 = g^z \), where \( y \overset{\text{\$}}{\leftarrow} \mathbb{Z}_p  \) 
            and \( z \overset{\text{\$}}{\leftarrow} \mathbb{Z}_p \).\\
            - During decryption, we have \( C_1^{\text{SK}} = (g^y)^x = g^{yx} \). Since \( y \) and \( z \) are independently and uniformly 
            chosen from \( \mathbb{Z}_p \)and group $\mathbb{Z}_p$ has prime order $p$, the probability that \( g^{yx} = g^z \) (i.e., \( C_1^{\text{SK}} = C_2 \)) 
            is $$\text{Pr}[ yx \equiv z  \text{ (mod } p )] = \frac{1}{p} $$
            - Thus, the probability of the decryption algorithm incorrectly return $0$ is \( \frac{1}{p} \), which is very small when the prime order 
            $p$ is sufficiently large.\\\\
            Therefore, we have shown that the correctness requirement of the scheme may sometimes not hold, but only with very small probability.
         \end{answer}
     \end{part}
\newpage
     \begin{part}
        \begin{answer}
            \begin{center}
            \begin{tabular}{|l|}
                \hline
                    \underline{\textbf{oracle} $\text{O}^{\text{DDH}_b[\mathbb{G},g]}$:} \\\\
                    \underline{private \textbf{procedure} Init($X,Y,Z$):} \;\;\;\;\;\; \textcolor{blue}{// $(X,Y,Z) \leftarrow \text{DDH}_b[\mathbb{G},g]$}\\
                        (PK, $Y$, $Z$) $\leftarrow (X,Y,Z)$\\
                        queried \(\leftarrow 0\)\\
                        \textbf{return} PK\\\\
                    \underline{public \textbf{procedure} Encrypt($M^0, M^1$):}  \;\;\;\;\;\; // \textcolor{blue}{$M^0,M^1 \in \mathcal{M} = \{0,1\}$}\\
                        \textbf{if} $|M^0| \ne |M^1|$ \textbf{or} queried $= 1$ \textbf{return} $\bot$\\
                        \textbf{if} $M^b = b$ \textbf{then}\\
                        \hspace*{22pt} $C \leftarrow (Y, Z)$ \;\;\;\;\;\; \textcolor{blue}{// When $M^0 = 0$, $C = (g^y, g^{xy})$. When $M^1 = 1$, $C = (g^y, g^{z})$}\\
                        \textbf{else if} $M^b = 0$ \\
                        \hspace*{22pt} $y' \overset{\text{\$}}{\leftarrow} \mathbb{Z}_p$\\
                        \hspace*{22pt} $C \leftarrow (g^{y'}, \text{PK}^{y'})$\;\;\;\;\;\; \textcolor{blue}{// Here $M^1 = 0$, $Z= g^z$, so we need a new $y'$ s.t.$C = (g^{y'}, g^{xy'})$ }\\
                        \textbf{else}\\
                        \hspace*{22pt} $z' \overset{\text{\$}}{\leftarrow} \mathbb{Z}_p$\\
                        \hspace*{22pt} $C \leftarrow (Y, g^{z'})$\;\;\;\;\;\; \textcolor{blue}{// Here $M^0 = 1$, $Z= g^{xy}$, so we need a new $z'$ s.t.$C = (g^{y}, g^{z'})$ }\\
                        queried $\leftarrow 1$\\
                        \textbf{return} $C$\\
                    \hline
            \end{tabular}
        \end{center}
         \end{answer}
     \end{part}

     \begin{part}
        \begin{answer}
             Using the oracle $\text{O}$ from part (b), for any distinguisher $D$, setting $D' = D^{\text{O}}$ we have:
                $$\text{Adv}^{\text{ddh}}_{\mathbb{G},g}(D) = \text{Adv}^{\text{1-ind-cpa}}_{\Pi}(D')$$
                Further, if $D$ is polynomial time, because O’s procedures all run in polynomial time, we also have that $D'$ is polynomial time.\\
                Then, because we assume DDH assumption holds for $\mathbb{G}$ with respect to $g$, $\text{Adv}^{\text{ddh}}_{\mathbb{G},g}(D)$ is negligible, which 
                means that $\text{Adv}^{\text{1-ind-cpa}}_{\Pi}(D')$ is also negligible (since the input of $D'$ first passed through the Init procedure of O, which takes 
                the output oracle $\text{DDH}_b[\mathbb{G}, g]$ as input).\\
                Therefore, we have shown that if DDH assumption holds for $\mathbb{G}$ with respect to $g$, then $\Pi$ is one-time IND-CPA secure.
         \end{answer}
     \end{part}

\end{question}

%%%%%%%%%%%%%%%%%%%%%%%% Task 3 %%%%%%%%%%%%%%%%%%%%%%%%
\begin{question}{Task 3 - Chosen-Ciphertext Security (10 points)} 
    \begin{part}
        \begin{answer}
             
         \end{answer}
     \end{part}

     \begin{part}
        \begin{answer}
             
         \end{answer}
     \end{part}

     \begin{part}
        \begin{answer}
             
         \end{answer}
     \end{part}

\end{question}

%%%%%%%%%%%%%%%%%%%%%%%% Task 4 %%%%%%%%%%%%%%%%%%%%%%%%
\begin{question}{Task 4 - AES-Based Signatures (15 points)} 
    \begin{part}
        \begin{answer}
             
         \end{answer}
     \end{part}

     \begin{part}
        \begin{answer}
             
         \end{answer}
     \end{part}

     \begin{part}
        \begin{answer}
             
         \end{answer}
     \end{part}

\end{question}
\end{document}