%%%%%%%%%%%%%%%%%%%%% PACKAGE IMPORTS %%%%%%%%%%%%%%%%%%%%%
\documentclass[11pt]{article}
\usepackage{amsmath, amsfonts, amsthm, amssymb}
\usepackage{lmodern}
\usepackage{microtype}
\usepackage{fullpage}       
\usepackage{changepage}
\usepackage{hyperref}
\usepackage{blindtext}
\hypersetup{
    colorlinks=true,
    linkcolor=blue,
    filecolor=magenta,      
    urlcolor=blue,
    pdftitle={Overleaf Example},
    pdfpagemode=FullScreen,
    }
\urlstyle{same}

\newenvironment{level}%
{\addtolength{\itemindent}{2em}}%
{\addtolength{\itemindent}{-2em}}

\usepackage{amsmath,amsthm,amssymb}


\usepackage[x11names, rgb]{xcolor}
\usepackage{graphicx}
\usepackage[nooldvoltagedirection]{circuitikz}
\usetikzlibrary{decorations,arrows,shapes}

\usepackage{datetime}
\usepackage{etoolbox}
\usepackage{enumerate}
\usepackage{enumitem}
\usepackage{listings}
\usepackage{array}
\usepackage{varwidth}
\usepackage{tcolorbox}
\usepackage{amsmath}
\usepackage{circuitikz}
\usepackage{verbatim}
\usepackage[linguistics]{forest}
\usepackage{listings}
\usepackage{xcolor}
\renewcommand{\rmdefault}{cmss}


\newcommand\doubleplus{+\kern-1.3ex+\kern0.8ex}
\newcommand\mdoubleplus{\ensuremath{\mathbin{+\mkern-10mu+}}}

\definecolor{codegreen}{rgb}{0,0.6,0}
\definecolor{codegray}{rgb}{0.5,0.5,0.5}
\definecolor{codepurple}{rgb}{0.58,0,0.82}
\definecolor{backcolour}{rgb}{0.95,0.95,0.92}

\lstdefinelanguage{JavaScript}{
  keywords={typeof, new, true, false, catch, function, return, null, catch, switch, var, if, in, while, do, else, case, break},
  keywordstyle=\color{blue}\bfseries,
  ndkeywords={class, export, boolean, throw, implements, import, this},
  ndkeywordstyle=\color{darkgray}\bfseries,
  identifierstyle=\color{black},
  sensitive=false,
  comment=[l]{//},
  morecomment=[s]{/*}{*/},
  commentstyle=\color{purple}\ttfamily,
  stringstyle=\color{red}\ttfamily,
  morestring=[b]',
  morestring=[b]"
}

\lstdefinestyle{mystyle}{
    language=JavaScript,
    backgroundcolor=\color{backcolour},   
    commentstyle=\color{codegreen},
    keywordstyle=\color{magenta},
    numberstyle=\tiny\color{codegray},
    stringstyle=\color{codepurple},
    basicstyle=\ttfamily\footnotesize,
    breakatwhitespace=false,         
    breaklines=true,                 
    captionpos=b,                    
    keepspaces=true,                 
    numbers=left,                    
    numbersep=5pt,                  
    showspaces=false,                
    showstringspaces=false,
    showtabs=false,                  
    tabsize=2
}

\lstset{style=mystyle}
\setlength{\parindent}{0pt}
\setlength{\parskip}{5pt plus 1pt}

\providetoggle{questionnumbers}
\settoggle{questionnumbers}{true}
\newcommand{\noquestionnumbers}{
    \settoggle{questionnumbers}{false}
}

\newcounter{questionCounter}
\newenvironment{question}[2][\arabic{questionCounter}]{%
    \ifnum\value{questionCounter}=0 \else {\newpage}\fi%
    \setcounter{partCounter}{0}%
    \vspace{.25in} \hrule \vspace{0.5em}%
    \noindent{\bf \iftoggle{questionnumbers}{Question #1: }{}#2}%
    \addtocounter{questionCounter}{1}%
    \vspace{0.8em} \hrule \vspace{.10in}%
}

\newcounter{partCounter}[questionCounter]
\renewenvironment{part}[1][\alph{partCounter}]{%
    \addtocounter{partCounter}{1}%
    \vspace{.10in}%
    \begin{indented}%
       {\bf (#1)} %
}{\end{indented}}

\def\indented#1{\list{}{}\item[]}
\let\indented=\endlist
\def\show#1{\ifdefempty{#1}{}{#1\\}}
\def\IMP{\rightarrow}
\def\AND{\wedge}
\def\OR{\vee}
\def\BI{\leftrightarrow}
\def\DIFF{\setminus}
\def\SUB{\subseteq}


\newcolumntype{C}{>{\centering\arraybackslash}m{1.5cm}}
\renewcommand\qedsymbol{$\blacksquare$}
\newtcolorbox{answer}
{
  colback   = green!5!white,    % Background colorucyitc,
  colframe  = green!75!black,   % Outline color
  box align = center,           % Align box on text line
  varwidth upper,               % Enables multi line input
  hbox                          % Bounds box to text width
}

\newcommand{\myhwname}{Homework 1}
\newcommand{\myname}{Sebastian Liu}
\newcommand{\myemail}{ll57@cs.washington.edu}
\newcommand{\mysection}{AB}
\newcommand{\dollararrow}{\stackrel{\$}{\leftarrow}}
%%%%%%%%%%%%%%%%%%%%%%%%%%%%%%%%%%%%%%%%%%%%%%%%%%%%%%%%%%%

%%%%%%%%%%%%%%%%%%% Document Options %%%%%%%%%%%%%%%%%%%%%%
\noquestionnumbers
%%%%%%%%%%%%%%%%%%%%%%%%%%%%%%%%%%%%%%%%%%%%%%%%%%%%%%%%%%%

%%%%%%%%%%%%%%%%%%%%%%%% WORK BELOW %%%%%%%%%%%%%%%%%%%%%%%%
\begin{document}

\begin{center}
    \textbf{Homework 2} \bigskip
\end{center}

%%%%%%%%%%%%%%%%%%%%%%%% Task 1 %%%%%%%%%%%%%%%%%%%%%%%%M
\begin{question}{Task 1 - Negligible Functions (10 points)}
    \begin{part}
       \begin{answer}
            \textbf{The function $f(k)$ is negligible.}\\\\
            \textbf{Proof.} For all $d \ge 1$, we want to find $k_0$ such that for all $k > k_0$,
            we have $k^{-log^2(k)} < k^{-d}$.\\
            We take logarithms to the base two of both sides, we get $k^{-log^2(k)} < k^{-d}$ is equivalent to:
            $$log^2(k) > d$$ 
            If we take for example $k_0 = 2^d$, then for all $k > k0$ we have:
            $$log^2(k) > log^2(k_0) = log^2(2^d) > d$$
            The first inequality follows from the fact that $k \mapsto log^2(k)$ grows monotonically, the last inequality follows from the fact that $log^2(2^d) = d^2 > d$ for all $d \ge 1$.
       \end{answer}
    \end{part}
\newpage
    \begin{part}
        \begin{answer}
            \begin{part}[i]
                    Since $f(k)$ and $g(k)$ are both negligible, by the definition of negligible functions, we know there exists $k_f$ and $k_g$ such that for all $k > k_f$ and $k > k_g$:
                $$f(k) < k^{-d} \text{ and } g(k) < k^{-d}$$
                We can take $k_1 =$ max$(k_f, k_g, 3)$, then for all $k > k_1$ we have:
                $$h_1(k) = f(k) + g(k) < 2k^{-d} < 2k^{-d-1} < k \cdot k^{-d - 1} < k^{-d}$$
                Therefore, by definition, $h_1(k)$ is negligible.\\
            \end{part}

            \begin{part}[ii]
                Assume $d' = d + c$, since $d \ge 1$ and $c > 0$, we know $d' > 1$. Since $f(k)$ is negligible, by the definition of negligible functions, we know for all $d' > 1$ there exists $k_f$ such that for all $k > k_f$: 
                $$f(k) < k^{-d'}$$
                We can take $k_2 = k_f$, then for all $k > k_2$ we have:
                $$h_2(k) = k^c \cdot f(k) < k^c \cdot k^{-d'} = k^{-d' + c} = k^{-(d + c) + c} = k^{-d}$$
                Therefore, by definition, $h_2(k)$ is negligible.\\
            \end{part}
        \end{answer}
     \end{part}
\end{question}

%%%%%%%%%%%%%%%%%%%%%%%% Task 2 %%%%%%%%%%%%%%%%%%%%%%%%
\begin{question}{Task 2 - Block Ciphers (10 points)}
    \begin{part}
       \begin{answer}
        Since $\Upsilon_0$ and $\Upsilon_1$ are evaluated using the same key, we know once we find a key that satisfies $E(K',0^n) = \Upsilon_0$, we can use the same key to satisfy $E(K',1^n) = \Upsilon_1$. \\
        We can see that the distinguisher only returns 1 when it finds a $K'$ that satisfy the condition above in the entire key space. Since the key for evaluating the block cipher is chosen uniformly at random and 
        the distinguisher is searching through the entire key space, we have:
            $$Pr[D^{KF[E]} \Rightarrow 1] = |K'| \times Pr[E(K',0^n) = \Upsilon_0] = 
            |K'| \times \underset{K \overset{\text{\$}}{\leftarrow} \{0,1\}^n}{\text{Pr}}[K = K'] = 2^n \times \frac{1}{2^{n}} = 1 $$
       \end{answer}
    \end{part}

    \begin{part}
        \begin{answer}
            For some $K' \in \{0,1\}^n$, we have:
            \begin{align*}
                & Pr[E(K', 0^n) = \Upsilon_0 \text{ and } E(K', 1^n) = \Upsilon_1] \\
                 =  &Pr[E(K', 0^n) = \Upsilon_0] \times Pr[E(K', 1^n) = \Upsilon_1 \mid E(K', 0^n) = \Upsilon_0]\\
                 = & \frac{1}{2^n} \times \frac{1}{2^n - 1} = \frac{1}{2^{2n} - 2^n} 
            \end{align*}
            Since distinguisher $D$ search through the entire key space, we have \textbf{the upper bound probability that $D$ outputs 1 is}:
            $$Pr[D^{RP[n]} \Rightarrow 1] = |K'| \times Pr[E(K', 0^n) = \Upsilon_0 \text{ and } E(K', 1^n) = \Upsilon_1] = 2^n \times \frac{1}{2^{2n} - 2^n} = \frac{1}{2^n -1}$$
            Therefore we know:
            $$\text{Adv}_\text{E}^{\text{prp}}(D) = |Pr[D^{KF[E]} \Rightarrow 1] - Pr[D^{RP[n] \Rightarrow 1}]| = |1 - \frac{1}{2^n -1}| = \frac{2^n - 2}{2^n -1}$$

        \end{answer}
    \end{part}

    \begin{part}
        \begin{answer}
           The distinguisher doesn't contradict the existence of secure pseudorandom permutations because the distinguisher runs in exponential time (since it searches through the entire key space which grows exponentially as the key length grows).
           The security of pseudorandom permutations is defined on distinguishers that runs in polynomial time, since in the real world we don't have a computer that has virtually unlimited computational power
           to efficiently run the $O(2^n)$ algorithm in the distinguisher above on a large enough key space. 
        \end{answer}
    \end{part}
   

\end{question}

%%%%%%%%%%%%%%%%%%%%%%%% Task 3 %%%%%%%%%%%%%%%%%%%%%%%%
\begin{question}{Task 3 - IND-CPA Security (9 points)} 
    \begin{part}
       \begin{answer}
        \begin{tabular}{|l|}
        \hline
            \underline{\textbf{distinguisher} $D^{\text{LR}_{b [\Pi]}}$:} \\
                $C_1 \leftarrow \text{LR}_{b [\Pi]}.\text{Encrypt}(0^n, 0^n)$ \\
                $C_2 \leftarrow \text{LR}_{b [\Pi]}.\text{Encrypt}(0^n, 1^n)$ \\
                \textbf{if $C_1 = C_2$ then} \\
                \hspace{1em} \textbf{return} 1 \\
                \textbf{return }0 \\
            \hline
            \end{tabular}\\
        Since Enc is deterministic, we know for some $K \overset{\text{\$}}{\leftarrow} $ Kg(): 
        $$Pr[D^{\text{LR}_{0 [\Pi]}} \Rightarrow 1] = Pr[\text{Enc}(K, 0^n) = \text{Enc}(K, 0^n)] = 1$$
        $$Pr[D^{\text{LR}_{1 [\Pi]}} \Rightarrow 1] = Pr[\text{Enc}(K, 0^n) = \text{Enc}(K, 1^n)] = 0$$
        Therefore, we know that $\Pi$ cannot be IND-CPA secure, since: 
        $$\text{Adv}_\Pi^{\text{ind-cpa}}(D) = |Pr[D^{\text{LR}_{0 [\Pi]}} \Rightarrow 1] - Pr[D^{\text{LR}_{1 [\Pi]}} \Rightarrow 1]| = |1 - 0| = 1$$
       \end{answer}
    \end{part}
\newpage
    \begin{part}
        \begin{answer}
            Assume we have some arbitrary plaintexts $M_0, M_1 \in \mathcal{M}$. \\\\
            For an arbitrary distinguisher, we know $Pr[D^{\text{LR}_{0 [\Pi]}} \Rightarrow 1]$ and $Pr[D^{\text{LR}_{1 [\Pi]}} \Rightarrow 1]$ only differ by
            the single query allowed on $\text{LR}_{0 [\Pi]}.\text{Encrypt}(M_0, M_1)$ and $\text{LR}_{1 [\Pi]}.\text{Encrypt}(M_0, M_1)$ respectively.\\
            
            By perfect secrecy, we know for all ciphertexts $C \in \mathcal{C}$:
            $$\underset{K \overset{\$}{\leftarrow} \text{Kg()}}{Pr}[\text{Enc}(K, M_0) = C] = \underset{K \overset{\$}{\leftarrow} \text{Kg()}}{Pr}[\text{Enc}(K, M_1) = C]$$
            Therefore, by the definition of $\text{LR}_{b [\Pi]}$, we know for all $M_0, M_1 \in \mathcal{M}, C \in \mathcal{C}$:
            \begin{align*}  
            & Pr[\text{LR}_{0 [\Pi]}.\text{Encrypt}(M_0, M_1) = C]  = \underset{K \overset{\$}{\leftarrow} \text{Kg()}}{Pr}[\text{Enc}(K, M_0) = C]\\
             = & \underset{K \overset{\$}{\leftarrow} \text{Kg()}}{Pr}[\text{Enc}(K, M_1) = C] = Pr[\text{LR}_{1 [\Pi]}.\text{Encrypt}(M_0, M_1) = C]
            \end{align*}
            Since now the only difference between  $Pr[D^{\text{LR}_{0 [\Pi]}} \Rightarrow 1]$ and $Pr[D^{\text{LR}_{1 [\Pi]}} \Rightarrow 1]$ are shown to have equal probability, we know:
            $$Pr[D^{\text{LR}_{0 [\Pi]}} \Rightarrow 1] = Pr[D^{\text{LR}_{1 [\Pi]}} \Rightarrow 1]$$
            Therefore:
            $$\text{Adv}_\Pi^{\text{ind-cpa}}(D) = |Pr[D^{\text{LR}_{0 [\Pi]}} \Rightarrow 1] - Pr[D^{\text{LR}_{1 [\Pi]}} \Rightarrow 1]| = 0$$
        \end{answer}
     \end{part}

     \begin{part}
        \begin{answer} 
        The one-time pad is deterministic, but its security comes from the randomness of the one-time key In a),
        the deterministic nature of one-time pad was exploitable because the same key was used for multiple encryptions, which allows patterns of the ciphertext
        to be leaked. In b), since we restrict the distinguisher to have only one query, which allowed the one-time pad to preserve its security properties by only allowing
        one key to be used once. b) does not contradict a), instead the combination of them shows that the one-time pad is (1,0)-IND-CPA secure.
        \end{answer}
     \end{part}

\end{question}

%%%%%%%%%%%%%%%%%%%%%%%% Task 4 %%%%%%%%%%%%%%%%%%%%%%%%
\begin{question}{Task 4 - More IND-CPA Security (8 points)}
    \begin{part}
       \begin{answer}
        \underline{\textbf{procedure} Dec'($K,C'$) :} \\
        \hspace{1em} $C \leftarrow (C'$ with the last occurrence of 0 removed) \\
        \textbf{return} Dec($K,C$) 
       \end{answer}
    \end{part}

    \begin{part}
        \begin{answer}
            
        \end{answer}
     \end{part}

\end{question}

%%%%%%%%%%%%%%%%%%%%%%%% Task 5 %%%%%%%%%%%%%%%%%%%%%%%%
\begin{question}{Task 5 - Pseudorandom Functions (8 points)}
    \begin{part}
       \begin{answer}
        
       \end{answer}
    \end{part}

    \begin{part}
        \begin{answer}
            
        \end{answer}
     \end{part}

\end{question}
\end{document}